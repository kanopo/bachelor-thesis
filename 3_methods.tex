

\section{Metodi}
\subsection{Dati}

Tutti i dati provengono da analisi ecografiche effettuate presso \textbf{Azienda Ospedaliero-Universitaria di Parma}.
Le immagini raccolte si rifanno al periodo compreso tra \textbf{Aprile 2022} e \textbf{Gennaio 2023} e sono state selezionando tenendo conto di alcuni parametri per standardizzare l'input come:
\begin{itemize}
  \item Indice di massa corporea (BMI)
  \item Età delle madri
  \item Problematiche durante la gravidanza
  \item Problematiche dopo il parto
  \item Immagini scattate con la medesima angolazione rispetto al femore
\end{itemize}


Le immagini hanno subito un processo di \textit{preprocessing} aggiuntivo per uniformare le dimensioni e la risoluzione. In particolare, sono state ridimensionate a immagini \textbf{$1280px$ di larcghezza} e \textbf{$876px$ di altezza}.


Inoltre data la tipologia del problema, si è scelto di convertire le immagine da \textbf{RGB} a immagini in \textbf{scala di grigi}.


Sono state realizzate manualmente delle \textbf{maschere} di segmantazione per ogni immagine, in modo da avere un \textit{ground truth} da confrontare con le predizioni del modello.


Data la scarsa quanti di dati a disposizione per l'addestramento della UNET, si è scelto di utilizzare alcune tecniche di \textit{data augmentation} per aumentare la quantità di dati a disposizione. In particolare si è scelto di utilizzare le seguenti tecniche applicate in modo casuale per ogni coppia \textbf{immagine-maschera}
\begin{itemize}
	\item \textit{Flip} orizzontale e verticale
	\item \textit{Rotazioni} di $35^{\circ}$
	\item \textit{Rumore} Gaussiano
\end{itemize}

Queste tecniche di \textit{data augmentation} migliorano notevolmente le segmentazioni ottenute mediante la rete UNET e rendono la rete più robusta a variazioni di luce e a rumore presente nelle immagini.


\subsection{Modello}

Il modello di partenza (\autoref{fig:unet}) è stato realizzato seguendo la struttura della rete \textbf{UNET} \cite{ronneberger2015unet} proposta da Olaf Ronneberger, Philipp Fischer e Thomas Brox nel 2015. La rete è stata implementata utilizzando il framework \textbf{PyTorch} \cite{pytorch}.

% TODO: Spiegone su UNET


