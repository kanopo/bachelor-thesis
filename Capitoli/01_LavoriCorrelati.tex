\chapter{Lavori correlati} \label{chap:related_works}
\section{Segmentazione} \label{sec:segmentazione} 

\begin{figure}[!ht]
  \begin{center}
    \includegraphics[width=0.4\textwidth]{Immagini/segmantion_example_image.png}
    \includegraphics[width=0.4\textwidth]{Immagini/segmantion_example_mask.png}
  \end{center}
  \caption{Segmentazione semantica}
  \label{fig:segmentazione}
\end{figure}


La segmentazione semantica rappresenta un campo di grande interesse e rilevanza nell'ambito
dell'elaborazione delle immagini e della visione artificiale. Questa tecnica si distingue per la sua
capacità di interpretare il contenuto delle immagini a un livello semantico, andando oltre la
semplice divisione dell'immagine in regioni omogenee basate su caratteristiche visive come il colore
o la texture. Specificamente, la segmentazione semantica mira a attribuire un'etichetta semantica a
ogni singolo pixel dell'immagine, permettendo così di identificare e categorizzare le diverse parti
che compongono la scena. L'obiettivo principale è fornire una comprensione approfondita del
contenuto visivo presente in un'immagine, traducendosi nella capacità di identificare e
categorizzare oggetti e regioni, il che rende possibile un'analisi dettagliata e una migliore
interpretazione dei dati visivi. Un esempio di applicazione della segmentazione semantica può essere
osservato nella figura \ref{fig:segmentazione}, dove l'obiettivo era quello di estrapolare le
informazioni relative al motociclista in una classe e le informazioni relative al veicolo in
un'altra, separando entrambe le classi dallo sfondo.



\section{Fully Convolutional Network} 
\begin{figure}[!ht]
  \begin{center}
    \includegraphics[width=0.8\textwidth]{Immagini/cnn.png}
  \end{center}
  \caption{CNN}
  \label{fig:cnn}
\end{figure}



L'articolo \textit{Fully Convolutional Networks for Semantic Segmentation} \cite{long2015fully}
introduce una tipologia di reti neurali convoluzionali (CNN) che, grazie all'assenza di layer
completamente connessi, permettono l'elaborazione di immagini di qualunque dimensione. Questo
approccio innovativo migliora notevolmente le capacità di apprendimento delle reti neurali,
consentendo la produzione di mappe di segmentazione più precise attraverso un apprendimento efficace
delle informazioni spaziali.

Le motivazioni per l'ampio utilizzo di queste reti nel settore della \textit{computer vision}
risiedono nell'assenza di strati completamente connessi, che solitamente vincolano l'ingresso a
dimensioni fisse per ogni immagine. Questa caratteristica permette alle reti di elaborare l'intera
immagine anziché solo frammenti, potenziando così l'apprendimento spaziale.

La maggiore flessibilità offerta da queste reti si traduce in un addestramento meno vincolato da
limitazioni dimensionali dell'input, risultando in una maggiore tolleranza agli errori e al rumore.
Di conseguenza, questa tipologia di reti si rivela particolarmente adatta in contesti caratterizzati
da una scarsità di dati.


\section{U-Net} \label{sec:unet} 

\begin{figure}[!ht]
  \begin{center}
    \includegraphics[width=0.8\textwidth]{Immagini/unet.png}
  \end{center}
  \caption{U-Net}\label{fig:unet}
\end{figure}


% \subsection{U-Net: Convolutional Networks for Biomedical Image Segmentation}
%
% Nel 2015, Olaf Ronneberger, Philipp Fischer e Thomas Brox hanno introdotto un nuovo modello di rete
% neurale convoluzionale chiamato \textit{U-Net} per la segmentazione semantica d'immagini biomedicali
% \cite{ronneberger2015unet}. Questa rete è stata progettata specificamente per affrontare le sfide
% associate alla segmentazione d'immagini biomedicali, come la necessità di segmentare strutture
% anatomiche precise con un numero limitato d'immagini di addestramento.
%
% Il modello \textit{U-Net} è caratterizzato da una struttura simmetrica, in cui la parte
% "contrattiva" (downsampling) cattura il contesto e la parte "espansiva" (upsampling) permette una
% localizzazione precisa. Questa struttura consente alla rete di combinare le informazioni di contesto
% con quelle locali, migliorando la precisione della segmentazione.
%
% Una delle principali innovazioni della \textit{U-Net} è l'introduzione di collegamenti a salti tra
% le parti contrattive ed espansive. Questi collegamenti trasferiscono le caratteristiche spaziali ad
% alta risoluzione dalla parte contrattiva a quella espansiva, consentendo una maggiore precisione
% nella localizzazione delle strutture segmentate.
%
% Il modello \textit{U-Net} ha dimostrato di ottenere risultati di segmentazione di alta qualità su
% diverse applicazioni biomedicali con un numero limitato d'immagini di addestramento, rendendolo uno
% strumento fondamentale per la segmentazione semantica in ambito biomedico.


\subsection{U-Net: Convolutional Networks for Biomedical Image Segmentation}

Nel 2015, Olaf Ronneberger, Philipp Fischer e Thomas Brox hanno introdotto il modello di rete
neurale convoluzionale \textit{U-Net} per la segmentazione semantica di immagini biomedicali
\cite{ronneberger2015unet}. Progettata specificatamente per affrontare le sfide della segmentazione
in ambito biomedico, come la necessità di segmentare con precisione strutture anatomiche con un
numero limitato di immagini di addestramento, U-Net ha rappresentato un notevole avanzamento.

Il modello \textit{U-Net} si distingue per una struttura simmetrica, dove la parte contrattiva
(downsampling) cattura il contesto e quella espansiva (upsampling) permette una localizzazione
precisa. Questa configurazione consente alla rete di fondere informazioni di contesto con quelle
locali, migliorando significativamente la precisione della segmentazione.

Una delle innovazioni chiave di \textit{U-Net} è l'introduzione di collegamenti a salti tra le parti
contrattive ed espansive, che trasferiscono caratteristiche spaziali ad alta risoluzione dalla parte
contrattiva a quella espansiva, aumentando la precisione nella localizzazione delle strutture
segmentate.

Grazie a queste caratteristiche, il modello \textit{U-Net} ha ottenuto risultati eccellenti nella
segmentazione di immagini biomedicali, anche con un numero limitato di immagini di addestramento,
diventando così un punto di riferimento nella segmentazione semantica biomedica.




\section{Segmentazione ossea} \label{sec:segmentazione_ossea}

Il lavoro \textit{Towards whole-body CT Bone Segmentation} \cite{10.1007/978-3-662-56537-7_59}
rappresenta un'importante analisi nello sviluppo di metodi e algoritmi avanzati per la segmentazione
ossea in immagini ottenute tramite tomografia computerizzata (TC) di tutto il corpo. Questo studio
pone l'accento sull'importanza della segmentazione ossea in campo medico, sia per la diagnosi di
condizioni patologiche sia per analisi dettagliate del tessuto osseo.

Il contributo fondamentale dell'articolo risiede nell'esplorazione di approcci innovativi e
nell'ottimizzazione di tecniche algoritmiche per identificare e isolare con precisione le strutture
ossee nelle immagini TC. Viene sottolineata l'importanza dell'uso di metodologie avanzate
nell'elaborazione delle immagini e dell'applicazione di algoritmi di visione artificiale e machine
learning per ottenere una segmentazione accurata.

L'articolo assume un ruolo significativo nel campo dell'informatica medica, evidenziando l'impiego
di soluzioni informatiche per un'analisi approfondita delle immagini mediche e riconoscendo
l'importanza delle tecniche di segmentazione ossea per applicazioni cliniche e di ricerca biomedica.


\section{Segmentazione di vasi sanguigni} \label{sec:segmentazione_vasi_sanguigni}

L'articolo \textit{Accurate Retinal Vessel Segmentation via Octave Convolution Neural Network}
\cite{fan2020accurate} introduce un approccio innovativo per la segmentazione precisa dei vasi
sanguigni retinici utilizzando le reti neurali a convoluzione ottava. Questa tecnica gioca un ruolo
cruciale nell'analisi delle immagini retiniche in ambito medico.

Il lavoro mette in luce i vantaggi offerti dalle reti neurali a convoluzione ottava, che utilizzano
differenti frequenze spaziali per catturare dettagli a diverse scale. Questo approccio risulta
particolarmente efficace nella segmentazione dei vasi sanguigni retinici, contribuendo
significativamente alla comprensione e diagnosi delle patologie oculari.

L'articolo dimostra l'efficacia di queste reti nel rilevare e isolare i vasi sanguigni della retina,
mostrando risultati più accurati rispetto ai metodi tradizionali. In definitiva, \textit{Accurate
Retinal Vessel Segmentation via Octave Convolution Neural Network} offre un contributo importante
nel campo della segmentazione vascolare retinica, sottolineando l'efficacia e l'importanza delle
reti neurali a convoluzione ottava nella diagnostica medica.
