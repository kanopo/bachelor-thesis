% !TeX spellcheck = it_IT

\chapter{Lavori correlati} \label{chap:related_works}


\section{Segmentazione} \label{sec:segmentazione} 
\todo{Aggiungere binary semantic segmentation}


\section{Fully Convolutional Network} 
\begin{figure}[H]
  \begin{center}
    \includegraphics[width=0.8\textwidth]{Immagini/cnn.png}
  \end{center}
  \caption{CNN}
  \label{fig:cnn}
\end{figure}



\label{sec:fcn} L'articolo \textit{Fully
Convolutional Networks for Semantic Segmentation} \cite{long2015fully} propone
l'utilizzo di una tipologia di reti neurali Convolutionali (CNN) che permettono
grazie all'assenza di layer completamente connessi di elaborare immagini di
qualunque dimensione. Questa nuova tipologia di reti migliora notevolmente le
le capacità di apprendimento delle reti neurali permettendo di produrre mappe
di segmentazione più precise grazie alla loro capacità di apprendimento di
informazioni spaziali.

Le motivazioni riguardanti l'ampio utilizzo nel settore della
\textit{computer vision} sono legate all'assenza di strati completamente
connessi (lineari) che vincolano l'input alla medesima grandezza per ogni
singola immagine, permettendo di fornire in input l'intera immagine e non
frammenti della stessa così da aumentare l'apprendimento spaziale della rete.

Questa maggior flessibilità comporta un'addestramento libero da limitaioni
sull'input comportando una maggiore tolleranza agli errori e al rumore
rendendo questa tipologia di reti particolarmente adatte a contesti poveri di
dati.




\section{U-Net} \label{sec:unet} \todo{Aggiungere unet}

\begin{figure}[H]
  \begin{center}
    \includegraphics[width=0.8\textwidth]{Immagini/unet.png}
  \end{center}
  \caption{U-Net}\label{fig:unet}
\end{figure}


L'architettura \textbf{U-net} \cite{ronneberger2015unet} è una particolare
implementazione di FCN che permette di effettuare segmentazioni semantiche,
in quanto è in grado di gestire input di qualsiasi dimensione e di produrre
mappe di segmentazione pi\`u precise grazie alla sua capacit\`a di apprendere
contesti spaziali.



\section{Segmentazione ossea} \label{sec:segmentazione_ossea}

Il lavoro \textit{Towards whole-body CT Bone Segmentation}
\cite{10.1007/978-3-662-56537-7_59} costituisce un'importante analisi volta a
sviluppare metodi e algoritmi avanzati per la segmentazione ossea in immagini
ottenute tramite tomografia computerizzata (TC) di tutto il corpo. Il documento
si concentra sull'importanza della segmentazione ossea nell'ambito medico per
diagnosticare condizioni patologiche e condurre analisi dettagliate del tessuto
osseo.

Il contributo principale dell'articolo consiste nella valutazione di approcci
innovativi e nell'ottimizzazione di tecniche algoritmiche per identificare e
isolare accuratamente le strutture ossee nelle immagini TC. Sottolinea
l'utilizzo di metodologie avanzate di elaborazione delle immagini e
l'applicazione di algoritmi di visione artificiale e machine learning per
ottenere una segmentazione precisa.

L'articolo è rilevante nell'ambito dell'informatica medica in quanto evidenzia
l'applicazione di soluzioni informatiche per l'analisi approfondita delle
immagini mediche, sottolineando l'importanza delle tecniche di segmentazione
ossea per fini clinici e di ricerca biomedica.


\section{Segmentazione di vasi sanguigni}
\label{sec:segmentazione_vasi_sanguigni}

L'articolo \textit{Accurate Retinal Vessel Segmentation via Octave Convolution
Neural Network} \cite{fan2020accurate} propone un approccio innovativo per la
segmentazione precisa dei vasi sanguigni retinici utilizzando le reti neurali a
convoluzione ottava. Questa segmentazione è un'importante fase nell'analisi
delle immagini retiniche in ambito medico.

L'articolo esamina il vantaggio delle reti neurali a convoluzione ottava, un
tipo di rete neurale che sfrutta differenti frequenze spaziali per catturare
dettagli a diverse scale. Questo approccio consente di migliorare la
segmentazione dei vasi sanguigni retinici, consentendo una migliore
comprensione e diagnosi di patologie oculari.

Il lavoro si concentra sull'efficacia delle reti neurali a convoluzione ottava
nel rilevare e isolare i vasi sanguigni della retina, evidenziando come questo
approccio abbia portato a risultati più accurati rispetto a metodi
convenzionali.

In conclusione, l'articolo \textit{Accurate Retinal Vessel Segmentation via
Octave Convolution Neural Network} costituisce un contributo significativo
nell'ambito della segmentazione vascolare retinica, evidenziando l'efficacia
delle reti neurali a convoluzione ottava e la loro importanza nella diagnostica
medica.

