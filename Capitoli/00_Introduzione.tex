% !TeX spellcheck = it_IT

\chapter{Introduzione} \label{chap:Introduzione}

La segmentazione semantica riveste un ruolo cruciale nell'ambito dell'analisi
delle immagini mediche, consentendo di identificare e isolare strutture
anatomiche di interesse. Questa tesi si concentra sull'applicazione di reti
neurali convoluzionali (CNN) e, in particolare, sull'utilizzo dell'architettura
U-Net per eseguire la segmentazione binaria di immagini ecografiche fetali al
fine di estrarre e delineare i femori.

Le immagini ecografiche fetali rappresentano una sfida complessa nell'ambito
della segmentazione, richiedendo un'accurata identificazione delle strutture
anatomiche, come i femori, per fini diagnostici e monitoraggio della crescita
fetale. La segmentazione binaria semantica si concentra sull'etichettare pixel
specifici dell'immagine associati ai femori, consentendo una comprensione
dettagliata delle strutture anatomiche in esame.

L'approccio adottato in questa tesi si basa sull'utilizzo della rete neurale
convoluzionale U-Net, una struttura architetturale nota per la sua efficacia
nella segmentazione di immagini biomediche. La peculiarità di U-Net risiede
nella sua capacità di catturare dettagli locali mantenendo, allo stesso tempo,
una visione globale dell'immagine, rendendola particolarmente adatta per
problemi di segmentazione dettagliata come l'estrazione dei femori dalle
ecografie fetali.

Attraverso l'analisi, l'implementazione e l'ottimizzazione di questa
architettura, il lavoro si propone di migliorare l'accuratezza e l'efficienza
della segmentazione, fornendo uno strumento affidabile per l'identificazione
automatica dei femori nelle immagini ecografiche fetali. L'obiettivo è quello di
apportare un contributo positivo all'avanzamento delle tecnologie di estrazione
delle informazioni dalle immagini ecografiche fetali, automatizzando e
facilitando una valutazione più precisa della crescita fetale allo scopo di
analizzare la densità minerale ossea fetale(BMD).
