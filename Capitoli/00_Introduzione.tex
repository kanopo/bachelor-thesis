\chapter{Introduzione}
\label{chap:Introduzione}


La segmentazione semantica gioca un ruolo cruciale nell'analisi delle immagini
mediche, consentendo di identificare e isolare strutture anatomiche di
interesse. Questa tesi si concentra sull'applicazione di reti neurali
convoluzionali (CNN), in particolare sull'utilizzo dell'architettura U-Net per
la segmentazione binaria di immagini ecografiche fetali, allo scopo di estrarre
e delineare i femori.


Le immagini ecografiche fetali rappresentano una sfida complessa nella
segmentazione, richiedendo un'accurata identificazione delle strutture
anatomiche. La segmentazione semantica binaria mira all'etichettatura
specifica di pixel associati ai femori, fornendo una comprensione dettagliata
delle strutture anatomiche esaminate.

L'approccio di questa tesi si basa sull'uso della rete neurale convoluzionale
U-Net, nota per la sua efficacia nella segmentazione di immagini biomediche. La
peculiarità di U-Net è la sua capacità di catturare dettagli locali pur
mantenendo una visione globale dell'immagine, rendendola particolarmente adatta
per la segmentazione dettagliata, come l'estrazione dei femori dalle ecografie
fetali.

Attraverso l'analisi, l'implementazione e l'ottimizzazione di questa
architettura, il lavoro mira a migliorare l'accuratezza e l'efficienza della
segmentazione, fornendo uno strumento affidabile per l'identificazione
automatica dei femori nelle immagini ecografiche fetali. L'obiettivo è di
apportare un contributo significativo all'avanzamento delle tecnologie di
estrazione delle informazioni dalle immagini ecografiche, automatizzando e
facilitando una valutazione più precisa della densità minerale ossea fetale
(BMD).

