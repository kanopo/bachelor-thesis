\chapter{Introduzione}
\label{chap:Introduzione}


La segmentazione semantica gioca un ruolo cruciale nell'analisi delle immagini
mediche, consentendo di identificare e isolare strutture anatomiche di
interesse. Questa tesi si concentra sull'applicazione di reti neurali
convoluzionali (CNN), in particolare sull'utilizzo dell'architettura U-Net per
la segmentazione binaria di immagini ecografiche fetali, allo scopo di estrarre
e delineare i femori.

\subsection{Sfida delle Immagini Ecografiche Fetali}

Le immagini ecografiche fetali rappresentano una sfida complessa nella
segmentazione, richiedendo un'accurata identificazione delle strutture
anatomiche. La segmentazione semantica binaria mira all'etichettatura
specifica di pixel associati ai femori, fornendo una comprensione dettagliata
delle strutture anatomiche esaminate.

\subsection{Utilizzo della Rete Neurale U-Net}
L'approccio di questa tesi si basa sull'uso della rete neurale convoluzionale
U-Net, nota per la sua efficacia nella segmentazione di immagini biomediche. La
peculiarità di U-Net è la sua capacità di catturare dettagli locali pur
mantenendo una visione globale dell'immagine, rendendola particolarmente adatta
per la segmentazione dettagliata, come l'estrazione dei femori dalle ecografie
fetali.

\subsection{Obiettivi e Contributi}
Attraverso l'analisi, l'implementazione e l'ottimizzazione di questa
architettura, il lavoro mira a migliorare l'accuratezza e l'efficienza della
segmentazione, fornendo uno strumento affidabile per l'identificazione
automatica dei femori nelle immagini ecografiche fetali. L'obiettivo è di
apportare un contributo significativo all'avanzamento delle tecnologie di
estrazione delle informazioni dalle immagini ecografiche, automatizzando e
facilitando una valutazione più precisa della densità minerale ossea fetale
(BMD).

% La segmentazione semantica riveste un ruolo cruciale nell'ambito dell'analisi
% delle immagini mediche, consentendo d'identificare e isolare strutture
% anatomiche d'interesse. Questa tesi si concentra sull'applicazione di reti
% neurali convoluzionali (CNN) e, in particolare, sull'utilizzo dell'architettura
% U-Net per eseguire la segmentazione binaria d'immagini ecografiche fetali al
% fine di estrarre e delineare i femori.
%
% Le immagini ecografiche fetali rappresentano una sfida complessa nell'ambito
% della segmentazione, richiedendo un'accurata identificazione delle strutture
% anatomiche, come i femori, per fini diagnostici e monitoraggio della crescita
% fetale. La segmentazione binaria semantica si concentra sull'etichettare pixel
% specifici dell'immagine associati ai femori, consentendo una comprensione
% dettagliata delle strutture anatomiche in esame.
%
% L'approccio adottato in questa tesi si basa sull'utilizzo della rete neurale
% convoluzionale U-Net, una struttura architetturale nota per la sua efficacia
% nella segmentazione d'immagini biomediche. La peculiarità di U-Net risiede
% nella sua capacità di catturare dettagli locali mantenendo, allo stesso tempo,
% una visione globale dell'immagine, rendendola particolarmente adatta per
% problemi di segmentazione dettagliata come l'estrazione dei femori dalle
% ecografie fetali.
%
% Attraverso l'analisi, l'implementazione e l'ottimizzazione di questa
% architettura, il lavoro si propone di migliorare l'accuratezza e l'efficienza
% della segmentazione, fornendo uno strumento affidabile per l'identificazione
% automatica dei femori nelle immagini ecografiche fetali. L'obiettivo è quello di
% apportare un contributo positivo all'avanzamento delle tecnologie di estrazione
% delle informazioni dalle immagini ecografiche fetali, automatizzando e
% facilitando una valutazione più precisa della crescita fetale allo scopo di
% analizzare la densità minerale ossea fetale(BMD).
