%TODO:
% Introduzione dove spieghi un po' in generale il problema 
% che hai affrontato (segmentazione automatica di un oggetto
% su immagini di ecografie) e l'obiettivo.


\section{Introduzione}


\subsection{Binary Semantic Segmentation}

La segmantazione semantica è una tecnica di \textit{computer vision} che permette di assegnare ad ogni pixel di un'immagine un'etichetta che ne descrive il contenuto.

Nello specifico in questa tesi si tratta una sottocategoria della semganzione semantica,
ovvero la \textbf{Binary Semantic Segmentation}(segmentazione semantica
binaria), questa tecnica di \textit{computer vision} permette di assegnare ad ogni pixel di un'immagine un'etichetta che ne descrive il contenuto, ma a differenza della segmentazione semantica classica, che permette di assegnare ad ogni pixel una delle $N$ possibili etichette, la segmentazione semantica binaria permette di assegnare ad ogni pixel una delle due etichette possibili: \textbf{oggetto} o \textbf{sfondo}.

La segmentazione \`e una tipologia di problema molto ricorrente in ambito medico, in quanto permette di automatizzare alcune procedure che altrimenti sarebbero eseguite manualmente, riducendo i tempi di esecuzione e i costi,
permettendo di ottenere risultati pi\`u precisi e accurati limitando lo sforzo umano.


\subsection{Fully Convolutional Network}

Le \textbf{Fully Convolutional Network} (FCN) \cite{long2015fully} sono una tipologia di reti neurali convoluzionali (CNN) che permettono di effettuare segmentazioni semantice, in quanto sono in grado di gestire input di qualsiasi dimensione e di produrre mappe di segmentazione pi\`u precise grazie alla loro capacit\`a di apprendere contesti spaziali.

Le motivazioni riguardanti l'ampio utilizzo nel settore della \textit{computer vision}
sono legate all'assenza di strati completamente connessi (o lineari), che permettono di gestire l'intera immagine in una sola volta, piuttosto che suddividerla in regioni e processarle separatamente, rendendole pi\`u efficienti in termini di tempo di elaborazione e risorse computazionali,
questa caratteristica comporta anche una maggiore flessibilit\`a dell'input, in quanto non sono presenti strati completamente connessi che bloccano le dimensioni dell'input, e una maggiore tolleranza agli errori e al rumore, in quanto sono in grado di gestire errori e rumore nei dati di input, rendendole robuste in situazioni reali.
% di questa tipologia di reti neurali convoluzionali ricadono nella flessibilit\`a delle informazioni che \`e in grado di gestire, nella sua capacit\`a di generalizzazione e nella tolleranza agli errori e al rumore.

Nello specifico \`e stata utilizzata una architettura di rete neurale che prende il nome di \textbf{U-Net} \cite{ronneberger2015unet},
la scelta \`e ricaduta su questa architettura in quanto \`e stata progettata per lavorare con un numero ridotto di immagini di addestramento e produrre segmentazioni pi\`u precise.

% he U-Net architecture stems from the so-called “fully convolutional network” proposed by Long, Shelhamer, and Darrell in 2014.[2]
%
% The main idea is to supplement a usual contracting network by successive layers, where pooling operations are replaced by upsampling operators. Hence these layers increase the resolution of the output. A successive convolutional layer can then learn to assemble a precise output based on this information.[1]
%
% One important modification in U-Net is that there are a large number of feature channels in the upsampling part, which allow the network to propagate context information to higher resolution layers. As a consequence, the expansive path is more or less symmetric to the contracting part, and yields a u-shaped architecture. The network only uses the valid part of each convolution without any fully connected layers.[2] To predict the pixels in the border region of the image, the missing context is extrapolated by mirroring the input image. This tiling strategy is important to apply the network to large images, since otherwise the resolution would be limited by the GPU memory. 

L'architettura di tipo U-net rappresenta lo standard di riferimento per la segmentazione semantica di immagini biomediche, in quanto \`e in grado di gestire un numero ridotto di immagini di addestramento e produrre segmentazioni pi\`u precise.


% La rete neurale UNET è una rete neurale convoluzionale (CNN) sviluppata per la segmentazione di immagini biomediche1
% . È stata progettata per lavorare con un numero ridotto di immagini di addestramento e produrre segmentazioni più precise3
% . L'architettura di UNET è simmetrica e composta da due parti principali: un percorso di contrazione (o encoder) e un percorso di espansione (o decoder)3
% . Il percorso di contrazione è costituito da una tipica rete convoluzionale, mentre il percorso di espansione è costituito da strati convoluzionali trasposti 2D1
% . UNET è un esempio di Fully Convolutional Network (FCN), che significa che non ha strati completamente connessi (o lineari)5
% .


% Flessibilità dell'input: A differenza delle reti neurali convenzionali che hanno strati completamente connessi (fully connected layers) che bloccano le dimensioni dell'input, le FCN possono gestire input di qualsiasi dimensione. Questo è dovuto all'eliminazione del dense layer in favore di strati convoluzionali 1x1, che permette di superare le limitazioni delle reti neurali convenzionali2
% .
% Riduzione della perdita di informazioni: Durante il processo di down-sampling nell'encoder, le reti FCN gestiscono la perdita di informazioni ricostruendo l'immagine e recuperando parte delle informazioni dai filtri di pooling prima della sintesi dalla feature map3
% .
% Efficienza: Le FCN sono in grado di gestire l'intera immagine in una sola volta, piuttosto che suddividerla in regioni e processarle separatamente. Questo le rende più efficienti in termini di tempo di elaborazione e risorse computazionali1
% .
% Precisione: Le FCN sono in grado di produrre mappe di segmentazione più precise, grazie alla loro capacità di apprendere contesti spaziali a livello locale e globale1
% .
% Adattabilità: Le FCN sono adattabili a una varietà di applicazioni, tra cui la segmentazione di immagini, la classificazione di immagini e l'object detection4
% .
% Tolleranza agli errori e al rumore: Le FCN sono in grado di gestire errori e rumore nei dati di input, rendendole robuste in situazioni reali6
% .
% Capacità di aggiornamento: Le FCN possono essere facilmente aggiornate con nuovi dati, permettendo un apprendimento continuo e l'adattamento a nuove informazioni6
% .
% Capacità di generalizzazione: Le FCN sono in grado di generalizzare bene da un set di addestramento a dati non visti, rendendole utili per una varietà di compiti di apprendimento automatico6
% .
% Indipendenza da assunzioni a priori: A differenza di molti altri modelli di apprendimento automatico, le FCN non richiedono assunzioni a priori sui dati6
% .
% Applicabilità a problemi di regressione e classificazione: Le FCN possono essere utilizzate sia per problemi di regressione che di classificazione, rendendole versatili per una varietà di compiti6
% .

% Le motivazioni che hanno portato alla scelta di una rete di questo tipo ricadono
% nella flessibilità delle informazioni che è in grado di gestire, nella sua
% capacità di generalizzazione e nella tolleranza agli errori e al rumore.
%
% Nello specifico \`e stata utilizzata una architettura di rete neurale
% che prende il nome di \textbf{U-Net}. \cite{ronneberger2015unet}
%
%
%



% Fully convolution networks
%
% A fully convolution network (FCN) is a neural network that only performs convolution (and subsampling or upsampling) operations. Equivalently, an FCN is a CNN without fully connected layers.
% Convolution neural networks
%
% The typical convolution neural network (CNN) is not fully convolutional because it often contains fully connected layers too (which do not perform the convolution operation), which are parameter-rich, in the sense that they have many parameters (compared to their equivalent convolution layers), although the fully connected layers can also be viewed as convolutions with kernels that cover the entire input regions, which is the main idea behind converting a CNN to an FCN. See this video by Andrew Ng that explains how to convert a fully connected layer to a convolutional layer.
% An example of an FCN
%
% An example of a fully convolutional network is the U-net (called in this way because of its U shape, which you can see from the illustration below), which is a famous network that is used for semantic segmentation, i.e. classify pixels of an image so that pixels that belong to the same class (e.g. a person) are associated with the same label (i.e. person), aka pixel-wise (or dense) classification.


% Convolutional networks are powerful visual models that
% yield hierarchies of features. We show that convolu-
% tional networks by themselves, trained end-to-end, pixels-
% to-pixels, exceed the state-of-the-art in semantic segmen-
% tation. Our key insight is to build “fully convolutional”
% networks that take input of arbitrary size and produce
% correspondingly-sized output with efficient inference and
% learning. We define and detail the space of fully convolu-
% tional networks, explain their application to spatially dense
% prediction tasks, and draw connections to prior models. We
% adapt contemporary classification networks (AlexNet [19],
% the VGG net [31], and GoogLeNet [32]) into fully convolu-
% tional networks and transfer their learned representations
% by fine-tuning [4] to the segmentation task. We then de-
% fine a novel architecture that combines semantic informa-
% tion from a deep, coarse layer with appearance information
% from a shallow, fine layer to produce accurate and detailed
% segmentations. Our fully convolutional network achieves
% state-of-the-art segmentation of PASCAL VOC (20% rela-
% tive improvement to 62.2% mean IU on 2012), NYUDv2,
% and SIFT Flow, while inference takes less than one fifth of a
% second for a typical image.


