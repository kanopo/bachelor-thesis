\section{Related Works}
\subsection{Segmentazione ossea}


Un progetto degno di nota \`e \textbf{Towards whole-body CT Bone Segmentation} \cite{Klein_2018}
poich\`e si propone di risolvere il problema della segmentazione ossea di umani e la raggiunge con
ottimi risultati con accuratezza del $96\%\pm 2\%$ mediante la metrica di \textbf{Dice Score} e $94\%\pm 2\%$ con la metrica di \textbf{Intersection over Union}.


Il progetto si basa su una rete neurale convoluzionale che utilizza l'architettura U-Net \cite{ronneberger2015unet}
e raggiunge ottimi risulati con circa 4000 immmagini e 60 epoche di training lasciando
l'architettura della rete invariata.


\subsection{Segmentazione vasi sanguigni}
L'articolo \textbf{Accurate Retinal Vessel Segmentation via
Octave Convolution Neural Network} \cite{fan2019octave} propone un metodo di segmentazione automatico per la segmentazione dei vasi sanguigni retinici.

L'implementazione del metodo \`e basata su una rete neurale convoluzionale che utilizza l'architettura Octave UNet, modello che segue l'architettura di U-Net \cite{ronneberger2015unet} ma utilizza l'operazione di convoluzione octave e delle convoluzioni octave trasposte.

Le convoluzini octave sono state introdotte in \textbf{Drop an Octave: Reducing Spatial Redundancy in Convolutional Neural Networks with Octave Convolution} \cite{chen2019drop} e sono una variante delle convoluzioni standard che prova a ovviare al problema di sbilanciamento delle \textit{features} all'interno delle mappe di \textit{features}.





